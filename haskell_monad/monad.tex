\documentclass[uplatex,a4paper,dvipdfmx]{jsarticle}
\usepackage{amsmath,amsfonts,amssymb,amsthm,bm,ascmac}
\usepackage{physics}
\usepackage{tikz}
\usepackage{gnuplot-lua-tikz,circuitikz}
\usetikzlibrary{intersections,calc,arrows.meta,cd,automata,positioning,shapes.geometric}
\usepackage{here,siunitx,multicol,multirow,systeme}
\usepackage[version=3]{mhchem}
\usepackage{chemfig,url,braket,enumerate,mathrsfs,otf,ulem,stmaryrd,titlesec}
\usepackage{listings}
\usepackage{bussproofs}
\usepackage{mathtools}
\usepackage{cite}
\usepackage{docmute}
% \lstset{
%     language={C++}, 
%     % backgroundcolor={\color[gray]{.85}},
%     basicstyle={\footnotesize\ttfamily},
%     identifierstyle={\footnotesize\ttfamily},
%     commentstyle={\footnotesize\ttfamily \color[rgb]{0.5,0.5,0.5}},
%     keywordstyle={\footnotesize\ttfamily \color[rgb]{0,0,0}},
%     ndkeywordstyle={\footnotesize\ttfamily},
%     stringstyle={\footnotesize\ttfamily}, 
%     frame={tb}, 
%     breaklines=true, 
%     columns=[l]{fullflexible},
%     numbers=left,
%     %numbers=none, 
%     xrightmargin=0zw,
%     xleftmargin=3zw, 
%     numberstyle={\scriptsize},
%     stepnumber=1, 
%     numbersep=1zw,
%     escapechar={\^},
%     morecomment=[l]{//}
% } 
\lstset{
    language={Haskell}, %プログラミング言語によって変える。
    basicstyle={\ttfamily\small},
    %% breaklines=true, %折り返し
}
\lstset{
    basicstyle={\ttfamily},
    identifierstyle={\small},
    commentstyle={\smallitshape},
    keywordstyle={\color{blue}\small\bfseries},
    commentstyle={\color{gray}\small},
    stringstyle={\color{red}\small},
    tabsize=2,
    frame={tb},
    breaklines=true,
    columns=[l]{fullflexible},
    numbers=left,
    xrightmargin=0zw,
    xleftmargin=3zw,
    numberstyle={\scriptsize},
    stepnumber=1,
    numbersep=1zw,
    lineskip=-0.5ex
}
\usepackage[dvipdfmx]{hyperref}
\usepackage{pxjahyper}
\hypersetup{%
 setpagesize=false,%
 bookmarks=true,%
 bookmarksdepth=tocdepth,%
 bookmarksnumbered=true,%
 colorlinks=true,%
 pdftitle={},%
 pdfsubject={},%
 pdfauthor={},%
 pdfkeywords={}}

\renewcommand{\labelenumi}{(\arabic{enumi})}
\DeclareMathOperator{\Sinarc}{\mathrm{Sin}^{-1}}
\DeclareMathOperator{\Cosarc}{\mathrm{Cos}^{-1}}
\DeclareMathOperator{\Tanarc}{\mathrm{Tan}^{-1}}
\DeclareMathOperator{\Real}{\mathbb{R}}
\DeclareMathOperator{\Complex}{\mathbb{C}}
\DeclareMathOperator{\Rational}{\mathbb{Q}}
\DeclareMathOperator{\Natural}{{\mathbb{N}}}
\DeclareMathOperator{\Integer}{\mathbb{Z}}
\DeclareMathOperator{\Ker}{\mathrm{Ker}}
\DeclareMathOperator{\diffset}{\backslash}
\DeclareMathOperator{\define}{\overset{\text{def}}{\Longleftrightarrow}}
\newcommand{\typed}{\!:\!}
\newcommand{\Var} {\mathord{\mathbf{Var}}}
\newcommand{\TVar}{\mathord{\mathbf{TVar}}}
\newcommand{\Type}{\mathord{\mathbf{Type}}}
\newcommand{\Term}{\mathord{\mathbf{\Lambda}}}
\newcommand{\Asn}{\mathord{\mathbf{Asn}}}
\newcommand{\Cont}{\mathord{\mathbf{Cont}}}
\newcommand{\Church}{\mathord{\mathbf{M}}}
\newcommand{\Sub}  {\mathord{\mathrm{Sub}}}
\newcommand{\FV}  {\mathord{\mathrm{FV}}}
\newcommand{\BV}  {\mathord{\mathrm{BV}}}
\newcommand{\vars}{\mathcal{V}}
\newcommand{\tvars}{\mathbb{V}}
\newcommand{\terms}{\mathcal{T}}
\newcommand{\types}{\mathbb{T}}
\newcommand{\contexts}{\mathfrak{G}}
\newcommand{\ID}  {\mathrm{ID}}
\newcommand{\WEAK}{\mathrm{WEAK}}
\newcommand{\EXC}{\mathrm{EXC}}
\newcommand{\VAR} {\mathrm{VAR}}
\newcommand{\APP} {\mathrm{APP}}
\newcommand{\BND} {\mathrm{BND}}
\newcommand{\INST}{\mathrm{INST}}
\newcommand{\GEN} {\mathrm{GEN}}
\newcommand{\tbeta}{\to_\beta}
\newcommand{\ttbeta}{\twoheadrightarrow_\beta}
\newcommand{\Typ}{\mathord{\mathrm{Typ}}}
\newcommand{\SN}{\mathsf{SN}}
\newcommand{\SAT}{\mathsf{SAT}}
\newcommand{\Bool}{{\mathbb{B}}}
\newcommand{\TCP}{\mathbf{TCP}}
\newcommand{\TYP}{\mathbf{TYP}}
\newcommand{\TI}{\mathrm{TI}}
\newcommand{\Inferencer}{\mathcal{I}^\TYP}
\newcommand{\SUP}{\mathbf{SUP}}
\newcommand{\Ltt}{\mathord{\mathtt{L}}}
\newcommand{\Rtt}{\mathord{\mathtt{R}}}
\newcommand{\Stt}{\mathord{\mathtt{S}}}
\newcommand{\Ktt}{\mathord{\mathtt{K}}}
\newcommand{\Itt}{\mathord{\mathtt{I}}}
% \newcommand{\init}{\mathrm{init}}
\newcommand{\init}{0}
\newcommand{\acc}{\mathrm{acc}}
\newcommand{\rej}{\mathrm{rej}}
\newcommand{\bak}{\mathrm{bak}}
\newcommand{\textspace}{\text{\textvisiblespace}\,}
\newcommand{\halts}{\!\downarrow}
\newcommand{\stalls}{\!\uparrow}
\newcommand{\pcto}{\rightharpoonup}
\newcommand{\problems}{\mathcal{P}_\Sigma}
\newcommand{\classified}{\in}
\newcommand{\Halt}{\mathtt{Halt}}
\newcommand{\Mcal}{\mathcal{M}}
\newcommand{\Ncal}{\mathcal{N}}
\newcommand{\id}{\mathrm{id}}
\newcommand{\Id}{\mathrm{Id}}
\newcommand{\Godel}{G\"{o}del{} }
\newcommand{\godel}{{\mathfrak{g}}}
\newcommand{\NJ}{\mathbf{NJ}}
\newcommand{\Ccal}{{\mathcal{C}}}
\newcommand{\Dcal}{{\mathcal{D}}}
\newcommand{\Fcal}{{\mathcal{F}}}
\newcommand{\Pcal}{{\mathcal{P}}}
\newcommand{\Ucal}{{\mathcal{U}}}
\newcommand{\Vcal}{{\mathcal{V}}}
\newcommand{\Wcal}{{\mathcal{W}}}
\newcommand{\Xbb}{{\mathbb{X}}}
\newcommand{\Ybb}{{\mathbb{Y}}}
\newcommand{\GCat}{{\mathcal G}}
\newcommand{\CSet}{{\mathord{\mathbf{Set}}}}
\newcommand{\End}{{\mathord{\mathbf{End}}}}
\newcommand{\Cat}{{\mathord{\mathbf{Cat}}}}
\newcommand{\CAT}{{\mathord{\mathbf{CAT}}}}
\newcommand{\Par}{{\mathord{\mathbf{Par}}}}
\newcommand{\Ibf}{{\mathord{\mathbf{I}}}}
\newcommand{\onebf}{{\mathord{\mathbf{1}}}}
\newcommand{\DP}{{\mathbf{DP}}}
\newcommand{\PP}{{\mathrm{P}}}
\newcommand{\NP}{{\mathrm{NP}}}
\newcommand{\Rone}{\mathbf{R1}}
\newcommand{\Rtwo}{\mathbf{R2}}
\newcommand{\Rthree}{\mathbf{R3}}
\newcommand{\Rfour}{\mathbf{R4}}
\newcommand{\Tot}{\mathsf{Tot}}
\newcommand{\Ocal}{\mathcal{O}}
\newcommand{\Kscr}{\mathscr{K}}
\newcommand{\Kcal}{\mathcal{K}}
\newcommand{\opposite}{\mathrm{op}}
\newcommand{\retract}{\mathrel{\triangleleft}}
\newcommand{\defined}{\mathord{\downarrow}}
\newcommand{\bindot}{\mathbin{\cdot}}
\newcommand{\Hask}{\mathsf{Hask}}

\DeclarePairedDelimiter{\interpret}{\llbracket}{\rrbracket}
\DeclarePairedDelimiter{\encode}{\langle}{\rangle}
\DeclarePairedDelimiter{\pair}{\lbrack}{\rbrack}
\DeclareMathOperator{\Hom}{\mathrm{Hom}}
\DeclareMathOperator{\ob}{\mathrm{ob}}
\DeclareMathOperator{\dom}{\mathrm{dom}}
\DeclareMathOperator{\cod}{\mathrm{cod}}
\DeclareMathOperator{\src}{\mathrm{src}}
\DeclareMathOperator{\tgt}{\mathrm{tgt}}
\DeclareMathOperator*{\plim}{plim}

\numberwithin{equation}{section}

\newcounter{general}

\newcommand{\gsection}[1]{\section{#1}\setcounter{general}{1}}

\theoremstyle{definition}
\newtheorem{dfnn}{定義}
\newtheorem{axiomn}{Axiom}
\newtheorem{thmn}{定理}
\newtheorem{propn}{命題}
\newtheorem{probn}{問題}
\newtheorem{lemn}{補題}
\newtheorem{corn}{系}
\newtheorem{examplen}{例}
\newtheorem{questionn}{問題}
\newtheorem{cautionn}{注意}
\newtheorem{remarksn}{注意}
\newtheorem{notationn}{記法}
\newtheorem{conjecturen}{予想}
\renewcommand{\thedfnn}{\arabic{general}}
\renewcommand{\theaxiomn}{\arabic{general}}
\renewcommand{\thethmn}{\arabic{general}}
\renewcommand{\thepropn}{\arabic{general}}
\renewcommand{\theprobn}{\arabic{general}}
\renewcommand{\thelemn}{\arabic{general}}
\renewcommand{\thecorn}{\arabic{general}}
\renewcommand{\theexamplen}{\arabic{general}}
\renewcommand{\thequestionn}{\arabic{general}}
\renewcommand{\thecautionn}{\arabic{general}}
\renewcommand{\theremarksn}{\arabic{general}}
\renewcommand{\thenotationn}{\arabic{general}}
\renewcommand{\theconjecturen}{\arabic{general}}

% \newenvironment{dfn}[1][]{\vspace{2ex}\begin{shadebox}\vspace{2ex}\begin{dfnn}[#1]}{\end{dfnn}\stepcounter{general}\vspace{2ex}\end{shadebox}\vspace{2ex}}
\newenvironment{dfn}[1][]{\begin{dfnn}[#1]}{\end{dfnn}\stepcounter{general}}
\newenvironment{axiom}[1][]{\begin{axiomn}[#1]}{\end{axiomn}\stepcounter{general}}
\newenvironment{thm}[1][]{\begin{thmn}[#1]}{\end{thmn}\stepcounter{general}}
\newenvironment{prop}[1][]{\begin{propn}[#1]}{\end{propn}\stepcounter{general}}
\newenvironment{prob}[1][]{\begin{probn}[#1]}{\end{propn}\stepcounter{general}}
\newenvironment{lem}[1][]{\begin{lemn}[#1]}{\end{lemn}\stepcounter{general}}
\newenvironment{cor}[1][]{\begin{corn}[#1]}{\end{corn}\stepcounter{general}}
\newenvironment{example}[1][]{\begin{examplen}[#1]}{\end{examplen}\stepcounter{general}}
\newenvironment{question}[1][]{\begin{questionn}[#1]}{\end{questionn}\stepcounter{general}}
\newenvironment{caution}[1][]{\begin{cautionn}[#1]}{\end{cautionn}\stepcounter{general}}
\newenvironment{remarks}[1][]{\begin{remarksn}[#1]}{\end{remarksn}\stepcounter{general}}
\newenvironment{notation}[1][]{\begin{notationn}[#1]}{\end{notationn}\stepcounter{general}}
\newenvironment{conjecture}[1][]{\begin{conjecturen}[#1]}{\end{conjecturen}\stepcounter{general}}

\title{モナドって何だよ}
\author{型推栄 (\texttt{@\_jj1lis\_uec})}

\begin{document}
\maketitle

\section{はじめに}

プログラミング言語におけるモナドって何? という直感が全然生えてこないので,まとめてみて理解を試みる.
よかったら読んでみてください.あと間違いがあればご指摘お願いします.

本稿では以下の知識を仮定する.
\begin{itemize}
    \item 圏論の入門書(Mac LaneとかLeinsterとか)の最初の数章を読んだことがある.
    \item Haskell について多少は知っている(でもモナドは知らない).
    \item 型について多少は知っている.
\end{itemize}
    
\section{圏論におけるモナド}
\setcounter{general}{1}

文献によって定義の揺れなどがあるだろうから,本稿で採用するモナドの定義を述べておく.

\begin{dfn}
    \label{dfn:monoidal_category}
    \begin{enumerate}
        \item $\Vcal$を圏,$I \in \Vcal$をその対象,$\otimes \colon \Vcal \times \Vcal \to \Vcal$を関手とする.
            $\otimes(A, B) \in \Vcal$は$A \otimes B$のようにも書く.
            次のような自然同型$\alpha, \lambda, \rho$が存在するとき,$\Vcal$は\textbf{モノイダル} (\textit{monoidal}) であると言う.
            なお以下では対象$I \in \Vcal$と関手$I \colon \mathbf{1} \to \Vcal$と同一視していることに注意せよ.
            \begin{equation*}
                \begin{array}{rccc}
                    \alpha \colon & \otimes \circ (\id_\Vcal \times \otimes) & \Rightarrow & \otimes \circ (\otimes \times \id_\Vcal), \\
                    \lambda \colon & \pi_2 & \Rightarrow & I \times \id_\Vcal, \\
                    \rho \colon & \pi_1 & \Rightarrow & \id_\Vcal \times I,
                \end{array}
            \end{equation*}
            \begin{equation} \label{equation:diagram_monoidal_isomorphisms}
                \begin{tikzcd}[sep=large]
                    \Vcal \times \Vcal \times \Vcal  \arrow[r, "\otimes \times \id_\Vcal"] \arrow[d, "\id_\Vcal \times \otimes"'] & \Vcal \times \Vcal \arrow[d, "\otimes"] \\
                    \Vcal \arrow[ur, Rightarrow, shorten=7mm, "\alpha" sloped]\times \Vcal \arrow[r, "\otimes"'] & \Vcal\rlap{,}
                \end{tikzcd}
                \qquad
                \begin{tikzcd}[sep=large]
                    \onebf \times \Vcal \arrow[r, "I \times \id_\Vcal"] \arrow[rd, bend right, "\pi_2"' name=X]
                        & \Vcal \times \Vcal \arrow[d, "\otimes"] \arrow[Rightarrow, from=X, shorten=5mm, "\lambda" sloped]
                        & \arrow[l, "\id_\Vcal \times I"'] \Vcal \times \onebf \arrow[ld, bend left, "\pi_1" name=Y]  \arrow[Rightarrow, from=Y, to=1-2, shorten=5mm, "\rho" sloped] \\
                        & \Vcal\rlap{.}
                \end{tikzcd}
            \end{equation}
            加えて,任意の$A, B, C, D \in \Vcal$に対して以下の図式は可換となる必要がある.
            ただしここで射$\alpha_{(A, B, C)}$を簡単に$\alpha_{A, B, C}$とも書いている:
            \begin{equation*}
                \begin{tikzcd}[row sep=large]
                    A \otimes (I \otimes B) \arrow[rr, "\alpha_{A, I, B}"] \arrow[rd, "\id_A \otimes \lambda_B"']
                        &
                        & (A \otimes I) \otimes B \arrow[ld, "\rho_A \otimes \id_B"] \\
                        & A \otimes B \rlap{,}
                \end{tikzcd}
            \end{equation*}
            \begin{equation*}
                \begin{tikzcd}[row sep=large, column sep=small]
                        & A \otimes (B \otimes (C \otimes D)) \arrow[ld, "\id_A \otimes \alpha_{B, C, D}"'] \arrow[rd, "\alpha_{A, B, C \otimes D}"]& \\
                    A \otimes ((B \otimes C) \otimes D) \arrow[d, "\alpha_{A, B \otimes C, D}"']
                        & 
                        & (A \otimes B) \otimes (C \otimes D) \arrow[d, "\alpha_{A \otimes B, C, D}"] \\
                        (A \otimes (B \otimes C)) \otimes D \arrow[rr, "\alpha_{A, B, C} \otimes \id_D"']
                        & 
                        & ((A \otimes B) \otimes C) \otimes D \rlap{.}
                \end{tikzcd}
            \end{equation*}
            このとき$I$を\textbf{単位対象} (\textit{unit object}),$\otimes$を\textbf{テンソル積} (\textit{tensor product}) と呼ぶ.
            また$\Vcal$を6つ組$(\Vcal, \otimes, I, \alpha, \lambda, \rho)$として明記する場合もある.
        % \item A monoidal category $\Vcal$ is \textit{symmetric} if $A \otimes B \cong B \otimes A$ holds by the following natural isomorphism $\sigma$
        %     such that $\sigma_{B, A} \circ \sigma_{A, B} = \id_{A \otimes B}$ for all objects $A, B \in \Vcal$.
        % \item A monoidal category is said to be \textit{strict} if Diagrams (\ref{equation:diagram_monoidal_isomorphisms}) are commutative.
        % \item A monoidal category $\Vcal$ is \textit{closed} if for each $A \in \Vcal$, the functor $A \otimes - \colon \Vcal \to \Vcal$ has a right adjoint, written as $A \multimap - \colon \Vcal \to \Vcal$.
        \item $(\Vcal, \otimes, I, \alpha, \lambda, \rho)$をモノイダル圏とする.$\Vcal$の\textbf{モノイド対象} (\textit{monoid object}) ないし単に\textbf{モノイド}とは,
            対象$M \in \Vcal$,\textbf{乗法} (\textit{multiplication}) と呼ばれる射$\mu \colon M \otimes M \to M$,\textbf{単位} (\textit{unit}) と呼ばれる射$\eta \colon I \to M$から成る
            3つ組$(M, \mu, \eta)$であって,以下の図式を可換にするもののことである.
            \begin{equation}
                \label{equation:monoid}
                \begin{tikzcd}[sep=large]
                    M \otimes M \otimes M \arrow[r, "\mu \otimes \id_M"] \arrow[d, "\id_M \otimes \mu"'] & M \otimes M \arrow[d, "\mu"] \\
                    M \otimes M \arrow[r, "\mu"'] & M\rlap{,}
                \end{tikzcd}
                \qquad
                \begin{tikzcd}[sep=large]
                    I \otimes M \arrow[r, "\eta \otimes \id_M"] \arrow[rd, bend right, "\lambda_M"' name=X]
                        & M \otimes M \arrow[d, "\mu"] 
                        & \arrow[l, "\id_M \otimes \eta"'] M \otimes I \arrow[ld, bend left, "\rho_M" name=Y] \\
                        & M\rlap{.}
                \end{tikzcd}
            \end{equation}
        \item  \textbf{モナド} (\textit{monad}) とは,自己関手の圏におけるモノイド対象である.
    \end{enumerate}
\end{dfn}

特にモナドについて補足しよう.
所与の圏$\Ccal$に対して,関手圏$[\Ccal, \Ccal]$---以後は$\End_\Ccal$と書く---は,テンソル積として関手の合成$\circ$,単位対象として恒等関手$\id_\Ccal$を取ることでモノイダル圏をなす.
以下の図式を見れば分かる通り,定義\ref{dfn:monoidal_category} (1) における自然変換$\alpha, \lambda, \rho$はすべて恒等自然変換となる.
\begin{equation*}
    % \begin{tikzcd}[sep=large]
    %     \End_\Ccal \times \End_\Ccal \times \End_\Ccal  \arrow[r, "\otimes \times \id_{\End_\Ccal}"] \arrow[d, "\id_{\End_\Ccal} \times \otimes"'] & \End_\Ccal \times \End_\Ccal \arrow[d, "\otimes"] \\
    %     \End_\Ccal \arrow[ur, Rightarrow, shorten=7mm, "\alpha" sloped]\times \End_\Ccal \arrow[r, "\otimes"'] & \End_\Ccal\rlap{,}
    % \end{tikzcd}
    \begin{tikzcd}[sep=large]
        (F, G, H) \arrow[r, mapsto, "\circ \times \id_{\End_\Ccal}"] \arrow[d, mapsto, "\id_{\End_\Ccal} \times \circ"'] & (F \circ G, H) \arrow[d, mapsto, "\circ"] \\
        (F, G \circ H) \arrow[r, mapsto, "\circ"', end anchor={[yshift=-2.9pt]}] & |[yshift=8pt]| \genfrac{}{}{0pt}{0}{(F \circ G) \circ H}{F \circ (G \circ H)\rlap{,}}
    \end{tikzcd}
    \qquad
    % \begin{tikzcd}[sep=large]
    %     \onebf \times \End_\Ccal \arrow[r, "\id_\Ccal \times \id_{\End_\Ccal}"] \arrow[rd, bend right, "\pi_2"' name=X]
    %                     & \End_\Ccal \times \End_\Ccal \arrow[d, "\circ"] \arrow[Rightarrow, from=X, shorten=5mm, "\lambda" sloped]
    %                     & \arrow[l, "\id_{\End_\Ccal} \times \id_\Ccal"'] \End_\Ccal \times \onebf \arrow[ld, bend left, "\pi_1" name=Y]  \arrow[Rightarrow, from=Y, to=1-2, shorten=5mm, "\rho" sloped] \\
    %                     & \End_\Ccal\rlap{.}
    % \end{tikzcd}
    \begin{tikzcd}[sep=large]
        (*, F) \arrow[r, mapsto, "\id_\Ccal \times \id_{\End_\Ccal}"] \arrow[rd, mapsto, bend right, "\pi_2"' name=X]
                        & (\id_\Ccal, F) \mid (G, \id_\Ccal) \arrow[d, mapsto, "\circ"] %\arrow[Rightarrow, from=X, shorten=5mm, "\lambda" sloped]
                        & \arrow[l, mapsto, "\id_{\End_\Ccal} \times \id_\Ccal"'] (G, *) \arrow[ld, mapsto, bend left, "\pi_1" name=Y] \\ % \arrow[Rightarrow, from=Y, to=1-2, shorten=5mm, "\rho" sloped] \\
                        & \genfrac{}{}{0pt}{0}{\id_\Ccal \circ F \mid G \circ \id_\Ccal}{F \mid G\rlap{.}}
    \end{tikzcd}
\end{equation*}
このとき$\End_\Ccal$の対象(すなわち関手)$T \colon \Ccal \to \Ccal$と
射(すなわち自然変換)$\mu \colon T \circ T \Rightarrow T$および$\eta \colon \id_\Ccal \Rightarrow T$
の組$(T, \mu, \eta)$は,以下の条件を満たすとき$\End_\Ccal$のモノイド対象となる.
\begin{align*}
    \mu \cdot (T\mu) &= \mu \cdot (\mu T), \\
    \mu \cdot (T\eta) &= \mu \cdot (\eta T) = \id_T.
\end{align*}
これはまさに,図式 \ref{equation:monoid} と同じことを言っているに過ぎない.
\begin{equation*}
    \begin{tikzcd}[sep=large]
        T \circ T \circ T \arrow[r, "\mu \circ \id_T"] \arrow[d, "\id_T \circ \mu"'] & T \circ T \arrow[d, "\mu"] \\
        T \circ T \arrow[r, "\mu"'] & T\rlap{,}
    \end{tikzcd}
    \qquad
    \begin{tikzcd}[sep=large]
        \id_\Ccal \circ T \arrow[r, "\eta \circ \id_T"] \arrow[rd, bend right, "\lambda_T"' name=X]
                        & T \circ T \arrow[d, "\mu"] 
                        & \arrow[l, "\id_T \circ \eta"'] T \circ \id_\Ccal \arrow[ld, bend left, "\rho_T" name=Y] \\
                        & T\rlap{.}
    \end{tikzcd}
\end{equation*}
ただしここで次のような記法を用いはじめた.
まず$\cdot$は$\End_\Ccal$における射の合成である.
また$T \mu$は各対象$A \in \Ccal$について
\begin{equation*}
    (T \mu)_A = T(\mu_A) \colon T \circ T \circ T(A) \to T \circ T(A)
\end{equation*}
とすることで定義される自然変換$T \circ T \circ T \Rightarrow T \circ T$を表す.
同様に $\mu T \colon T \circ T \circ T \Rightarrow T \circ T$で$(\mu T)_A = \mu_{T(A)}$により定まる自然変換を表す.
この$(T, \mu, \eta)$が$\Ccal$におけるモナドである.

\section{Haskellにおけるモナド}

ようやく本題に入ろう.関数型プログラミング言語は数多あるが,本稿ではHaskellを用いて説明を行う.
Haskellの型を対象,関数を射と見て,その全体を$\Hask$と書くことにする.実は$\Hask$は圏である.

\begin{remarks}
\end{remarks}

\end{document}
