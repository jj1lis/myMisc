\usepackage{amsmath,amsfonts,amssymb,amsthm,bm,ascmac}
\usepackage{physics}
\usepackage{tikz}
\usepackage{gnuplot-lua-tikz,circuitikz}
\usetikzlibrary{intersections,calc,arrows.meta,cd,automata,positioning,shapes.geometric}
\usepackage{here,siunitx,multicol,multirow,systeme}
\usepackage[version=3]{mhchem}
\usepackage{chemfig,url,braket,enumerate,mathrsfs,otf,ulem,stmaryrd,titlesec}
\usepackage{listings}
\usepackage{bussproofs}
\usepackage{mathtools}
\usepackage{cite}
\usepackage{docmute}
% \lstset{
%     language={C++}, 
%     % backgroundcolor={\color[gray]{.85}},
%     basicstyle={\footnotesize\ttfamily},
%     identifierstyle={\footnotesize\ttfamily},
%     commentstyle={\footnotesize\ttfamily \color[rgb]{0.5,0.5,0.5}},
%     keywordstyle={\footnotesize\ttfamily \color[rgb]{0,0,0}},
%     ndkeywordstyle={\footnotesize\ttfamily},
%     stringstyle={\footnotesize\ttfamily}, 
%     frame={tb}, 
%     breaklines=true, 
%     columns=[l]{fullflexible},
%     numbers=left,
%     %numbers=none, 
%     xrightmargin=0zw,
%     xleftmargin=3zw, 
%     numberstyle={\scriptsize},
%     stepnumber=1, 
%     numbersep=1zw,
%     escapechar={\^},
%     morecomment=[l]{//}
% } 
\lstset{
    language={Haskell}, %プログラミング言語によって変える。
    basicstyle={\ttfamily\small},
    %% breaklines=true, %折り返し
}
\lstset{
    basicstyle={\ttfamily},
    identifierstyle={\small},
    commentstyle={\smallitshape},
    keywordstyle={\color{blue}\small\bfseries},
    commentstyle={\color{gray}\small},
    stringstyle={\color{red}\small},
    tabsize=2,
    frame={tb},
    breaklines=true,
    columns=[l]{fullflexible},
    numbers=left,
    xrightmargin=0zw,
    xleftmargin=3zw,
    numberstyle={\scriptsize},
    stepnumber=1,
    numbersep=1zw,
    lineskip=-0.5ex
}
\usepackage[dvipdfmx]{hyperref}
\usepackage{pxjahyper}
\hypersetup{%
 setpagesize=false,%
 bookmarks=true,%
 bookmarksdepth=tocdepth,%
 bookmarksnumbered=true,%
 colorlinks=true,%
 pdftitle={},%
 pdfsubject={},%
 pdfauthor={},%
 pdfkeywords={}}

\renewcommand{\labelenumi}{(\arabic{enumi})}
\DeclareMathOperator{\Sinarc}{\mathrm{Sin}^{-1}}
\DeclareMathOperator{\Cosarc}{\mathrm{Cos}^{-1}}
\DeclareMathOperator{\Tanarc}{\mathrm{Tan}^{-1}}
\DeclareMathOperator{\Real}{\mathbb{R}}
\DeclareMathOperator{\Complex}{\mathbb{C}}
\DeclareMathOperator{\Rational}{\mathbb{Q}}
\DeclareMathOperator{\Natural}{{\mathbb{N}}}
\DeclareMathOperator{\Integer}{\mathbb{Z}}
\DeclareMathOperator{\Ker}{\mathrm{Ker}}
\DeclareMathOperator{\diffset}{\backslash}
\DeclareMathOperator{\define}{\overset{\text{def}}{\Longleftrightarrow}}
\newcommand{\typed}{\!:\!}
\newcommand{\Var} {\mathord{\mathbf{Var}}}
\newcommand{\TVar}{\mathord{\mathbf{TVar}}}
\newcommand{\Type}{\mathord{\mathbf{Type}}}
\newcommand{\Term}{\mathord{\mathbf{\Lambda}}}
\newcommand{\Asn}{\mathord{\mathbf{Asn}}}
\newcommand{\Cont}{\mathord{\mathbf{Cont}}}
\newcommand{\Church}{\mathord{\mathbf{M}}}
\newcommand{\Sub}  {\mathord{\mathrm{Sub}}}
\newcommand{\FV}  {\mathord{\mathrm{FV}}}
\newcommand{\BV}  {\mathord{\mathrm{BV}}}
\newcommand{\vars}{\mathcal{V}}
\newcommand{\tvars}{\mathbb{V}}
\newcommand{\terms}{\mathcal{T}}
\newcommand{\types}{\mathbb{T}}
\newcommand{\contexts}{\mathfrak{G}}
\newcommand{\ID}  {\mathrm{ID}}
\newcommand{\WEAK}{\mathrm{WEAK}}
\newcommand{\EXC}{\mathrm{EXC}}
\newcommand{\VAR} {\mathrm{VAR}}
\newcommand{\APP} {\mathrm{APP}}
\newcommand{\BND} {\mathrm{BND}}
\newcommand{\INST}{\mathrm{INST}}
\newcommand{\GEN} {\mathrm{GEN}}
\newcommand{\tbeta}{\to_\beta}
\newcommand{\ttbeta}{\twoheadrightarrow_\beta}
\newcommand{\Typ}{\mathord{\mathrm{Typ}}}
\newcommand{\SN}{\mathsf{SN}}
\newcommand{\SAT}{\mathsf{SAT}}
\newcommand{\Bool}{{\mathbb{B}}}
\newcommand{\TCP}{\mathbf{TCP}}
\newcommand{\TYP}{\mathbf{TYP}}
\newcommand{\TI}{\mathrm{TI}}
\newcommand{\Inferencer}{\mathcal{I}^\TYP}
\newcommand{\SUP}{\mathbf{SUP}}
\newcommand{\Ltt}{\mathord{\mathtt{L}}}
\newcommand{\Rtt}{\mathord{\mathtt{R}}}
\newcommand{\Stt}{\mathord{\mathtt{S}}}
\newcommand{\Ktt}{\mathord{\mathtt{K}}}
\newcommand{\Itt}{\mathord{\mathtt{I}}}
% \newcommand{\init}{\mathrm{init}}
\newcommand{\init}{0}
\newcommand{\acc}{\mathrm{acc}}
\newcommand{\rej}{\mathrm{rej}}
\newcommand{\bak}{\mathrm{bak}}
\newcommand{\textspace}{\text{\textvisiblespace}\,}
\newcommand{\halts}{\!\downarrow}
\newcommand{\stalls}{\!\uparrow}
\newcommand{\pcto}{\rightharpoonup}
\newcommand{\problems}{\mathcal{P}_\Sigma}
\newcommand{\classified}{\in}
\newcommand{\Halt}{\mathtt{Halt}}
\newcommand{\Mcal}{\mathcal{M}}
\newcommand{\Ncal}{\mathcal{N}}
\newcommand{\id}{\mathrm{id}}
\newcommand{\Id}{\mathrm{Id}}
\newcommand{\Godel}{G\"{o}del{} }
\newcommand{\godel}{{\mathfrak{g}}}
\newcommand{\NJ}{\mathbf{NJ}}
\newcommand{\Ccal}{{\mathcal{C}}}
\newcommand{\Dcal}{{\mathcal{D}}}
\newcommand{\Fcal}{{\mathcal{F}}}
\newcommand{\Pcal}{{\mathcal{P}}}
\newcommand{\Ucal}{{\mathcal{U}}}
\newcommand{\Vcal}{{\mathcal{V}}}
\newcommand{\Wcal}{{\mathcal{W}}}
\newcommand{\Xbb}{{\mathbb{X}}}
\newcommand{\Ybb}{{\mathbb{Y}}}
\newcommand{\GCat}{{\mathcal G}}
\newcommand{\CSet}{{\mathord{\mathbf{Set}}}}
\newcommand{\End}{{\mathord{\mathbf{End}}}}
\newcommand{\Cat}{{\mathord{\mathbf{Cat}}}}
\newcommand{\CAT}{{\mathord{\mathbf{CAT}}}}
\newcommand{\Par}{{\mathord{\mathbf{Par}}}}
\newcommand{\Ibf}{{\mathord{\mathbf{I}}}}
\newcommand{\onebf}{{\mathord{\mathbf{1}}}}
\newcommand{\DP}{{\mathbf{DP}}}
\newcommand{\PP}{{\mathrm{P}}}
\newcommand{\NP}{{\mathrm{NP}}}
\newcommand{\Rone}{\mathbf{R1}}
\newcommand{\Rtwo}{\mathbf{R2}}
\newcommand{\Rthree}{\mathbf{R3}}
\newcommand{\Rfour}{\mathbf{R4}}
\newcommand{\Tot}{\mathsf{Tot}}
\newcommand{\Ocal}{\mathcal{O}}
\newcommand{\Kscr}{\mathscr{K}}
\newcommand{\Kcal}{\mathcal{K}}
\newcommand{\opposite}{\mathrm{op}}
\newcommand{\retract}{\mathrel{\triangleleft}}
\newcommand{\defined}{\mathord{\downarrow}}
\newcommand{\bindot}{\mathbin{\cdot}}
\newcommand{\Hask}{\mathsf{Hask}}

\DeclarePairedDelimiter{\interpret}{\llbracket}{\rrbracket}
\DeclarePairedDelimiter{\encode}{\langle}{\rangle}
\DeclarePairedDelimiter{\pair}{\lbrack}{\rbrack}
\DeclareMathOperator{\Hom}{\mathrm{Hom}}
\DeclareMathOperator{\ob}{\mathrm{ob}}
\DeclareMathOperator{\dom}{\mathrm{dom}}
\DeclareMathOperator{\cod}{\mathrm{cod}}
\DeclareMathOperator{\src}{\mathrm{src}}
\DeclareMathOperator{\tgt}{\mathrm{tgt}}
\DeclareMathOperator*{\plim}{plim}

\numberwithin{equation}{section}

\newcounter{general}

\newcommand{\gsection}[1]{\section{#1}\setcounter{general}{1}}

\theoremstyle{definition}
\newtheorem{dfnn}{定義}
\newtheorem{axiomn}{Axiom}
\newtheorem{thmn}{定理}
\newtheorem{propn}{命題}
\newtheorem{probn}{問題}
\newtheorem{lemn}{補題}
\newtheorem{corn}{系}
\newtheorem{examplen}{例}
\newtheorem{questionn}{問題}
\newtheorem{cautionn}{注意}
\newtheorem{remarksn}{注意}
\newtheorem{notationn}{記法}
\newtheorem{conjecturen}{予想}
\renewcommand{\thedfnn}{\arabic{general}}
\renewcommand{\theaxiomn}{\arabic{general}}
\renewcommand{\thethmn}{\arabic{general}}
\renewcommand{\thepropn}{\arabic{general}}
\renewcommand{\theprobn}{\arabic{general}}
\renewcommand{\thelemn}{\arabic{general}}
\renewcommand{\thecorn}{\arabic{general}}
\renewcommand{\theexamplen}{\arabic{general}}
\renewcommand{\thequestionn}{\arabic{general}}
\renewcommand{\thecautionn}{\arabic{general}}
\renewcommand{\theremarksn}{\arabic{general}}
\renewcommand{\thenotationn}{\arabic{general}}
\renewcommand{\theconjecturen}{\arabic{general}}

% \newenvironment{dfn}[1][]{\vspace{2ex}\begin{shadebox}\vspace{2ex}\begin{dfnn}[#1]}{\end{dfnn}\stepcounter{general}\vspace{2ex}\end{shadebox}\vspace{2ex}}
\newenvironment{dfn}[1][]{\begin{dfnn}[#1]}{\end{dfnn}\stepcounter{general}}
\newenvironment{axiom}[1][]{\begin{axiomn}[#1]}{\end{axiomn}\stepcounter{general}}
\newenvironment{thm}[1][]{\begin{thmn}[#1]}{\end{thmn}\stepcounter{general}}
\newenvironment{prop}[1][]{\begin{propn}[#1]}{\end{propn}\stepcounter{general}}
\newenvironment{prob}[1][]{\begin{probn}[#1]}{\end{propn}\stepcounter{general}}
\newenvironment{lem}[1][]{\begin{lemn}[#1]}{\end{lemn}\stepcounter{general}}
\newenvironment{cor}[1][]{\begin{corn}[#1]}{\end{corn}\stepcounter{general}}
\newenvironment{example}[1][]{\begin{examplen}[#1]}{\end{examplen}\stepcounter{general}}
\newenvironment{question}[1][]{\begin{questionn}[#1]}{\end{questionn}\stepcounter{general}}
\newenvironment{caution}[1][]{\begin{cautionn}[#1]}{\end{cautionn}\stepcounter{general}}
\newenvironment{remarks}[1][]{\begin{remarksn}[#1]}{\end{remarksn}\stepcounter{general}}
\newenvironment{notation}[1][]{\begin{notationn}[#1]}{\end{notationn}\stepcounter{general}}
\newenvironment{conjecture}[1][]{\begin{conjecturen}[#1]}{\end{conjecturen}\stepcounter{general}}
