\documentclass[uplatex,a4paper,dvipdfmx]{jsarticle}
\usepackage{amsmath,amsfonts,amssymb,amsthm,bm}
\usepackage{physics}
\usepackage[top=15truemm,bottom=20truemm,left=15truemm,right=15truemm]{geometry}
\usepackage{tikz}
\usepackage{gnuplot-lua-tikz}
\usetikzlibrary{intersections,calc,arrows.meta}
\usepackage{here}
\usepackage{siunitx}
\usepackage{multicol}
\usepackage{systeme}
\usepackage[version=3]{mhchem}
\usepackage{chemfig}
\usepackage{url}
\usepackage{braket}
\renewcommand{\thefootnote}{[\arabic{footnote}]}
\renewcommand{\labelenumi}{(\arabic{enumi})}
\newcommand{\Sinarc}{\mathrm{Sin}^{-1}}
\newcommand{\Cosarc}{\mathrm{Cos}^{-1}}
\newcommand{\Tanarc}{\mathrm{Tan}^{-1}}

\DeclareMathOperator*{\map}{map}
\DeclareMathOperator*{\fix}{fix}
\newcommand{\append}{\hspace{1pt} \tilde{+} \hspace{1pt}}

\title{}
\author{}

\begin{document}
    ある集合$G$に対して、その要素を用いて作成した要素$n$個の重複順列$P$すべての集合$S_P$を返すサブルーチン$f_G:n \mapsto S_P$を考える。

    $f$の再帰的定義は

    \begin{equation*}
        f_G := \lambda n. \left\{
            \begin{array}{lc}
                \displaystyle \left( \map_{f(G,n-1) \to p} p \append G_1,\, \map_{f(G,n-1) \to p} p \append G_2,\, \ldots \,,\,\map_{f(G,n-1) \to p} p \append G_{\left| G \right|} \right) & n > 1 \\
                \left(G_1, G_2, \ldots, G_{\left| G \right|}\right) & n = 1
            \end{array}
            \right.
    \end{equation*}
    である。ただし$G_k$は$G$の要素に重複なく振り分けた番号である。以下の通り演算を定義した。

    \begin{table}[H]
        \centering
        \begin{tabular}{c|l} \hline
            演算 & 定義 \\ \hline
            $\displaystyle \map_{X \to x} \langle \text{expr} \rangle$ & $X$の各要素$x$に対して操作$\langle \text{expr} \rangle$を行う。$X$が順列の場合は元の順序を破壊しない。 \\ \hline
            $P \append x$ & 順列$P$の末尾に要素$x$を結合する。 \\ \hline
        \end{tabular}
    \end{table}

    また、不動点コンビネータを$\fix$とする。

    ところで一般に、ある再帰関数$f(x)$を$f(x) = U(f,x)$という形で表す関数$U$が存在する。さらに$U$を$V: f \mapsto U(f,x)$で定義するとき、
    \begin{equation}
        f(x) = \fix(V) \tag{Lem.}
    \end{equation}
    が成り立つ。

    今回の例に当てはめると、

    \begin{align*}
        U &= \lambda f_G. \lambda n. \left\{
            \begin{array}{lc}
                \displaystyle \left( \map_{f(G,n-1) \to p} p \append G_1,\, \map_{f(G,n-1) \to p} p \append G_2,\, \ldots \,,\,\map_{f(G,n-1) \to p} p \append G_{\left| G \right|} \right) & n > 1 \\
                \left(G_1, G_2, \ldots, G_{\left| G \right|}\right) & n = 1
            \end{array}
            \right. \\
        V &= \lambda f_G. \left\{
            \begin{array}{lc}
                \lambda n.\displaystyle \left( \map_{f(G,n-1) \to p} p \append G_1,\, \map_{f(G,n-1) \to p} p \append G_2,\, \ldots \,,\,\map_{f(G,n-1) \to p} p \append G_{\left| G \right|} \right) & n > 1 \\
                \lambda n. \left(G_1, G_2, \ldots, G_{\left| G \right|}\right) & n = 1
            \end{array}
            \right. \\
            &= \lambda f_Gn. \left\{
            \begin{array}{lc}
                \displaystyle \left( \map_{f(G,n-1) \to p} p \append G_1,\, \map_{f(G,n-1) \to p} p \append G_2,\, \ldots \,,\,\map_{f(G,n-1) \to p} p \append G_{\left| G \right|} \right) & n > 1 \\
                \left(G_1, G_2, \ldots, G_{\left| G \right|}\right) & n = 1
            \end{array}
            \right.
    \end{align*}

    Order:
    \begin{align*}
        n \sum_{k=0}^{n} |G|\cdot|G|^k &= n \sum_{k=1}^{n+1} |G|^k \\
            &= n \frac{|G|(|G|^{n+1} - 1)}{|G|-1} \\
            &= \frac{n(|G|^{n+2}) - n|G|}{|G|-1} \\
        \therefore\hspace{5pt} &O(n|G|^{n+1})
    \end{align*}
\end{document}
